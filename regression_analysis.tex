% Options for packages loaded elsewhere
\PassOptionsToPackage{unicode}{hyperref}
\PassOptionsToPackage{hyphens}{url}
\documentclass[
]{article}
\usepackage{xcolor}
\usepackage[margin=1in]{geometry}
\usepackage{amsmath,amssymb}
\setcounter{secnumdepth}{5}
\usepackage{iftex}
\ifPDFTeX
  \usepackage[T1]{fontenc}
  \usepackage[utf8]{inputenc}
  \usepackage{textcomp} % provide euro and other symbols
\else % if luatex or xetex
  \usepackage{unicode-math} % this also loads fontspec
  \defaultfontfeatures{Scale=MatchLowercase}
  \defaultfontfeatures[\rmfamily]{Ligatures=TeX,Scale=1}
\fi
\usepackage{lmodern}
\ifPDFTeX\else
  % xetex/luatex font selection
\fi
% Use upquote if available, for straight quotes in verbatim environments
\IfFileExists{upquote.sty}{\usepackage{upquote}}{}
\IfFileExists{microtype.sty}{% use microtype if available
  \usepackage[]{microtype}
  \UseMicrotypeSet[protrusion]{basicmath} % disable protrusion for tt fonts
}{}
\makeatletter
\@ifundefined{KOMAClassName}{% if non-KOMA class
  \IfFileExists{parskip.sty}{%
    \usepackage{parskip}
  }{% else
    \setlength{\parindent}{0pt}
    \setlength{\parskip}{6pt plus 2pt minus 1pt}}
}{% if KOMA class
  \KOMAoptions{parskip=half}}
\makeatother
\usepackage{color}
\usepackage{fancyvrb}
\newcommand{\VerbBar}{|}
\newcommand{\VERB}{\Verb[commandchars=\\\{\}]}
\DefineVerbatimEnvironment{Highlighting}{Verbatim}{commandchars=\\\{\}}
% Add ',fontsize=\small' for more characters per line
\usepackage{framed}
\definecolor{shadecolor}{RGB}{248,248,248}
\newenvironment{Shaded}{\begin{snugshade}}{\end{snugshade}}
\newcommand{\AlertTok}[1]{\textcolor[rgb]{0.94,0.16,0.16}{#1}}
\newcommand{\AnnotationTok}[1]{\textcolor[rgb]{0.56,0.35,0.01}{\textbf{\textit{#1}}}}
\newcommand{\AttributeTok}[1]{\textcolor[rgb]{0.13,0.29,0.53}{#1}}
\newcommand{\BaseNTok}[1]{\textcolor[rgb]{0.00,0.00,0.81}{#1}}
\newcommand{\BuiltInTok}[1]{#1}
\newcommand{\CharTok}[1]{\textcolor[rgb]{0.31,0.60,0.02}{#1}}
\newcommand{\CommentTok}[1]{\textcolor[rgb]{0.56,0.35,0.01}{\textit{#1}}}
\newcommand{\CommentVarTok}[1]{\textcolor[rgb]{0.56,0.35,0.01}{\textbf{\textit{#1}}}}
\newcommand{\ConstantTok}[1]{\textcolor[rgb]{0.56,0.35,0.01}{#1}}
\newcommand{\ControlFlowTok}[1]{\textcolor[rgb]{0.13,0.29,0.53}{\textbf{#1}}}
\newcommand{\DataTypeTok}[1]{\textcolor[rgb]{0.13,0.29,0.53}{#1}}
\newcommand{\DecValTok}[1]{\textcolor[rgb]{0.00,0.00,0.81}{#1}}
\newcommand{\DocumentationTok}[1]{\textcolor[rgb]{0.56,0.35,0.01}{\textbf{\textit{#1}}}}
\newcommand{\ErrorTok}[1]{\textcolor[rgb]{0.64,0.00,0.00}{\textbf{#1}}}
\newcommand{\ExtensionTok}[1]{#1}
\newcommand{\FloatTok}[1]{\textcolor[rgb]{0.00,0.00,0.81}{#1}}
\newcommand{\FunctionTok}[1]{\textcolor[rgb]{0.13,0.29,0.53}{\textbf{#1}}}
\newcommand{\ImportTok}[1]{#1}
\newcommand{\InformationTok}[1]{\textcolor[rgb]{0.56,0.35,0.01}{\textbf{\textit{#1}}}}
\newcommand{\KeywordTok}[1]{\textcolor[rgb]{0.13,0.29,0.53}{\textbf{#1}}}
\newcommand{\NormalTok}[1]{#1}
\newcommand{\OperatorTok}[1]{\textcolor[rgb]{0.81,0.36,0.00}{\textbf{#1}}}
\newcommand{\OtherTok}[1]{\textcolor[rgb]{0.56,0.35,0.01}{#1}}
\newcommand{\PreprocessorTok}[1]{\textcolor[rgb]{0.56,0.35,0.01}{\textit{#1}}}
\newcommand{\RegionMarkerTok}[1]{#1}
\newcommand{\SpecialCharTok}[1]{\textcolor[rgb]{0.81,0.36,0.00}{\textbf{#1}}}
\newcommand{\SpecialStringTok}[1]{\textcolor[rgb]{0.31,0.60,0.02}{#1}}
\newcommand{\StringTok}[1]{\textcolor[rgb]{0.31,0.60,0.02}{#1}}
\newcommand{\VariableTok}[1]{\textcolor[rgb]{0.00,0.00,0.00}{#1}}
\newcommand{\VerbatimStringTok}[1]{\textcolor[rgb]{0.31,0.60,0.02}{#1}}
\newcommand{\WarningTok}[1]{\textcolor[rgb]{0.56,0.35,0.01}{\textbf{\textit{#1}}}}
\usepackage{graphicx}
\makeatletter
\newsavebox\pandoc@box
\newcommand*\pandocbounded[1]{% scales image to fit in text height/width
  \sbox\pandoc@box{#1}%
  \Gscale@div\@tempa{\textheight}{\dimexpr\ht\pandoc@box+\dp\pandoc@box\relax}%
  \Gscale@div\@tempb{\linewidth}{\wd\pandoc@box}%
  \ifdim\@tempb\p@<\@tempa\p@\let\@tempa\@tempb\fi% select the smaller of both
  \ifdim\@tempa\p@<\p@\scalebox{\@tempa}{\usebox\pandoc@box}%
  \else\usebox{\pandoc@box}%
  \fi%
}
% Set default figure placement to htbp
\def\fps@figure{htbp}
\makeatother
\setlength{\emergencystretch}{3em} % prevent overfull lines
\providecommand{\tightlist}{%
  \setlength{\itemsep}{0pt}\setlength{\parskip}{0pt}}
\usepackage{booktabs}
\usepackage{longtable}
\usepackage{array}
\usepackage{multirow}
\usepackage{wrapfig}
\usepackage{float}
\usepackage{colortbl}
\usepackage{pdflscape}
\usepackage{tabu}
\usepackage{threeparttable}
\usepackage{threeparttablex}
\usepackage[normalem]{ulem}
\usepackage{makecell}
\usepackage{xcolor}
\usepackage{bookmark}
\IfFileExists{xurl.sty}{\usepackage{xurl}}{} % add URL line breaks if available
\urlstyle{same}
\hypersetup{
  pdftitle={Multiple Regression Analysis: Boston Housing Dataset},
  pdfauthor={Karen Monie},
  hidelinks,
  pdfcreator={LaTeX via pandoc}}

\title{Multiple Regression Analysis: Boston Housing Dataset}
\author{Karen Monie}
\date{2026-01-17}

\begin{document}
\maketitle

{
\setcounter{tocdepth}{2}
\tableofcontents
}
\section{Conclusion}\label{conclusion}

This multiple regression analysis of the Boston Housing Dataset reveals
several important relationships between neighborhood characteristics and
median home values. The model explains a substantial portion of the
variance in housing values, indicating that the selected predictors
capture significant variation in property values.

Key findings from the analysis include:

\begin{itemize}
\tightlist
\item
  The model is statistically significant overall, as indicated by the
  F-statistic
\item
  Several predictor variables show significant relationships with median
  home values
\item
  Diagnostic plots help assess whether the regression assumptions are
  met
\item
  The relationship between predictors and the response variable can be
  visualized through scatter plots
\end{itemize}

The diagnostic plots provide insights into model adequacy: - The fitted
vs actual plot shows how well the model predicts housing values -
Residual plots help identify potential violations of regression
assumptions - Q-Q plots assess the normality of residuals

This analysis provides a foundation for understanding the factors that
influence housing values in the Boston area and can be extended with
additional model specifications, transformations, or variable selection
techniques.

\section{Introduction}\label{introduction}

This report presents a multiple regression analysis of the Boston
Housing Dataset. The objective is to model the relationship between the
median value of owner-occupied homes (medv) and various neighborhood
characteristics. Multiple regression allows us to examine how several
predictor variables simultaneously influence housing values, providing
insights into the factors that contribute to property valuation in the
Boston area.

The analysis includes: - Data exploration and summary statistics - Model
specification and estimation - Model diagnostics to assess assumptions -
Visualization of relationships and model fit - Interpretation of results

\section{Data Loading and
Exploration}\label{data-loading-and-exploration}

\begin{Shaded}
\begin{Highlighting}[]
\CommentTok{\# Load required libraries}
\FunctionTok{library}\NormalTok{(ggplot2)}
\FunctionTok{library}\NormalTok{(knitr)}
\FunctionTok{library}\NormalTok{(kableExtra)}

\CommentTok{\# Load the Boston Housing Dataset}
\NormalTok{boston }\OtherTok{\textless{}{-}} \FunctionTok{read.csv}\NormalTok{(}\StringTok{"Boston Housing Dataset.csv"}\NormalTok{, }\AttributeTok{stringsAsFactors =} \ConstantTok{FALSE}\NormalTok{)}

\CommentTok{\# Remove the first column (row index) if it exists}
\ControlFlowTok{if}\NormalTok{(}\FunctionTok{names}\NormalTok{(boston)[}\DecValTok{1}\NormalTok{] }\SpecialCharTok{==} \StringTok{"X"} \SpecialCharTok{||} \FunctionTok{names}\NormalTok{(boston)[}\DecValTok{1}\NormalTok{] }\SpecialCharTok{==} \StringTok{""}\NormalTok{) \{}
\NormalTok{  boston }\OtherTok{\textless{}{-}}\NormalTok{ boston[, }\SpecialCharTok{{-}}\DecValTok{1}\NormalTok{]}
\NormalTok{\}}

\CommentTok{\# Display basic information about the dataset}
\FunctionTok{cat}\NormalTok{(}\StringTok{"Dataset Dimensions:"}\NormalTok{, }\FunctionTok{dim}\NormalTok{(boston)[}\DecValTok{1}\NormalTok{], }\StringTok{"observations,"}\NormalTok{, }\FunctionTok{dim}\NormalTok{(boston)[}\DecValTok{2}\NormalTok{], }\StringTok{"variables}\SpecialCharTok{\textbackslash{}n}\StringTok{"}\NormalTok{)}
\end{Highlighting}
\end{Shaded}

\begin{verbatim}
## Dataset Dimensions: 506 observations, 14 variables
\end{verbatim}

\begin{Shaded}
\begin{Highlighting}[]
\FunctionTok{cat}\NormalTok{(}\StringTok{"}\SpecialCharTok{\textbackslash{}n}\StringTok{Variable Names:}\SpecialCharTok{\textbackslash{}n}\StringTok{"}\NormalTok{)}
\end{Highlighting}
\end{Shaded}

\begin{verbatim}
## 
## Variable Names:
\end{verbatim}

\begin{Shaded}
\begin{Highlighting}[]
\FunctionTok{print}\NormalTok{(}\FunctionTok{names}\NormalTok{(boston))}
\end{Highlighting}
\end{Shaded}

\begin{verbatim}
##  [1] "crim"    "zn"      "indus"   "chas"    "nox"     "rm"      "age"    
##  [8] "dis"     "rad"     "tax"     "ptratio" "black"   "lstat"   "medv"
\end{verbatim}

\begin{Shaded}
\begin{Highlighting}[]
\CommentTok{\# Display summary statistics}
\NormalTok{summary\_stats }\OtherTok{\textless{}{-}} \FunctionTok{summary}\NormalTok{(boston)}
\FunctionTok{print}\NormalTok{(summary\_stats)}
\end{Highlighting}
\end{Shaded}

\begin{verbatim}
##       crim                zn             indus            chas        
##  Min.   : 0.00632   Min.   :  0.00   Min.   : 0.46   Min.   :0.00000  
##  1st Qu.: 0.08205   1st Qu.:  0.00   1st Qu.: 5.19   1st Qu.:0.00000  
##  Median : 0.25651   Median :  0.00   Median : 9.69   Median :0.00000  
##  Mean   : 3.61352   Mean   : 11.36   Mean   :11.14   Mean   :0.06917  
##  3rd Qu.: 3.67708   3rd Qu.: 12.50   3rd Qu.:18.10   3rd Qu.:0.00000  
##  Max.   :88.97620   Max.   :100.00   Max.   :27.74   Max.   :1.00000  
##       nox               rm             age              dis        
##  Min.   :0.3850   Min.   :3.561   Min.   :  2.90   Min.   : 1.130  
##  1st Qu.:0.4490   1st Qu.:5.886   1st Qu.: 45.02   1st Qu.: 2.100  
##  Median :0.5380   Median :6.208   Median : 77.50   Median : 3.207  
##  Mean   :0.5547   Mean   :6.285   Mean   : 68.57   Mean   : 3.795  
##  3rd Qu.:0.6240   3rd Qu.:6.623   3rd Qu.: 94.08   3rd Qu.: 5.188  
##  Max.   :0.8710   Max.   :8.780   Max.   :100.00   Max.   :12.127  
##       rad              tax           ptratio          black       
##  Min.   : 1.000   Min.   :187.0   Min.   :12.60   Min.   :  0.32  
##  1st Qu.: 4.000   1st Qu.:279.0   1st Qu.:17.40   1st Qu.:375.38  
##  Median : 5.000   Median :330.0   Median :19.05   Median :391.44  
##  Mean   : 9.549   Mean   :408.2   Mean   :18.46   Mean   :356.67  
##  3rd Qu.:24.000   3rd Qu.:666.0   3rd Qu.:20.20   3rd Qu.:396.23  
##  Max.   :24.000   Max.   :711.0   Max.   :22.00   Max.   :396.90  
##      lstat            medv      
##  Min.   : 1.73   Min.   : 5.00  
##  1st Qu.: 6.95   1st Qu.:17.02  
##  Median :11.36   Median :21.20  
##  Mean   :12.65   Mean   :22.53  
##  3rd Qu.:16.95   3rd Qu.:25.00  
##  Max.   :37.97   Max.   :50.00
\end{verbatim}

\begin{Shaded}
\begin{Highlighting}[]
\CommentTok{\# Display first few rows}
\FunctionTok{cat}\NormalTok{(}\StringTok{"}\SpecialCharTok{\textbackslash{}n}\StringTok{First 6 rows of the dataset:}\SpecialCharTok{\textbackslash{}n}\StringTok{"}\NormalTok{)}
\end{Highlighting}
\end{Shaded}

\begin{verbatim}
## 
## First 6 rows of the dataset:
\end{verbatim}

\begin{Shaded}
\begin{Highlighting}[]
\FunctionTok{kable}\NormalTok{(}\FunctionTok{head}\NormalTok{(boston), }\AttributeTok{caption =} \StringTok{"First 6 observations of the Boston Housing Dataset"}\NormalTok{) }\SpecialCharTok{\%\textgreater{}\%}
  \FunctionTok{kable\_styling}\NormalTok{(}\AttributeTok{latex\_options =} \FunctionTok{c}\NormalTok{(}\StringTok{"striped"}\NormalTok{, }\StringTok{"hold\_position"}\NormalTok{), }\AttributeTok{font\_size =} \DecValTok{9}\NormalTok{)}
\end{Highlighting}
\end{Shaded}

\begingroup\fontsize{9}{11}\selectfont

\begin{longtable}[t]{rrrrrrrrrrrrrr}
\caption{\label{tab:data-structure}First 6 observations of the Boston Housing Dataset}\\
\toprule
crim & zn & indus & chas & nox & rm & age & dis & rad & tax & ptratio & black & lstat & medv\\
\midrule
\cellcolor{gray!10}{0.00632} & \cellcolor{gray!10}{18} & \cellcolor{gray!10}{2.31} & \cellcolor{gray!10}{0} & \cellcolor{gray!10}{0.538} & \cellcolor{gray!10}{6.575} & \cellcolor{gray!10}{65.2} & \cellcolor{gray!10}{4.0900} & \cellcolor{gray!10}{1} & \cellcolor{gray!10}{296} & \cellcolor{gray!10}{15.3} & \cellcolor{gray!10}{396.90} & \cellcolor{gray!10}{4.98} & \cellcolor{gray!10}{24.0}\\
0.02731 & 0 & 7.07 & 0 & 0.469 & 6.421 & 78.9 & 4.9671 & 2 & 242 & 17.8 & 396.90 & 9.14 & 21.6\\
\cellcolor{gray!10}{0.02729} & \cellcolor{gray!10}{0} & \cellcolor{gray!10}{7.07} & \cellcolor{gray!10}{0} & \cellcolor{gray!10}{0.469} & \cellcolor{gray!10}{7.185} & \cellcolor{gray!10}{61.1} & \cellcolor{gray!10}{4.9671} & \cellcolor{gray!10}{2} & \cellcolor{gray!10}{242} & \cellcolor{gray!10}{17.8} & \cellcolor{gray!10}{392.83} & \cellcolor{gray!10}{4.03} & \cellcolor{gray!10}{34.7}\\
0.03237 & 0 & 2.18 & 0 & 0.458 & 6.998 & 45.8 & 6.0622 & 3 & 222 & 18.7 & 394.63 & 2.94 & 33.4\\
\cellcolor{gray!10}{0.06905} & \cellcolor{gray!10}{0} & \cellcolor{gray!10}{2.18} & \cellcolor{gray!10}{0} & \cellcolor{gray!10}{0.458} & \cellcolor{gray!10}{7.147} & \cellcolor{gray!10}{54.2} & \cellcolor{gray!10}{6.0622} & \cellcolor{gray!10}{3} & \cellcolor{gray!10}{222} & \cellcolor{gray!10}{18.7} & \cellcolor{gray!10}{396.90} & \cellcolor{gray!10}{5.33} & \cellcolor{gray!10}{36.2}\\
\addlinespace
0.02985 & 0 & 2.18 & 0 & 0.458 & 6.430 & 58.7 & 6.0622 & 3 & 222 & 18.7 & 394.12 & 5.21 & 28.7\\
\bottomrule
\end{longtable}
\endgroup{}

\begin{Shaded}
\begin{Highlighting}[]
\CommentTok{\# Check for missing values}
\NormalTok{missing\_values }\OtherTok{\textless{}{-}} \FunctionTok{colSums}\NormalTok{(}\FunctionTok{is.na}\NormalTok{(boston))}
\ControlFlowTok{if}\NormalTok{(}\FunctionTok{sum}\NormalTok{(missing\_values) }\SpecialCharTok{\textgreater{}} \DecValTok{0}\NormalTok{) \{}
  \FunctionTok{cat}\NormalTok{(}\StringTok{"}\SpecialCharTok{\textbackslash{}n}\StringTok{Missing values per variable:}\SpecialCharTok{\textbackslash{}n}\StringTok{"}\NormalTok{)}
  \FunctionTok{print}\NormalTok{(missing\_values[missing\_values }\SpecialCharTok{\textgreater{}} \DecValTok{0}\NormalTok{])}
\NormalTok{\} }\ControlFlowTok{else}\NormalTok{ \{}
  \FunctionTok{cat}\NormalTok{(}\StringTok{"}\SpecialCharTok{\textbackslash{}n}\StringTok{No missing values found in the dataset.}\SpecialCharTok{\textbackslash{}n}\StringTok{"}\NormalTok{)}
\NormalTok{\}}
\end{Highlighting}
\end{Shaded}

\begin{verbatim}
## 
## No missing values found in the dataset.
\end{verbatim}

\section{Model Specification}\label{model-specification}

The multiple regression model is specified as:

\[medv = \beta_0 + \beta_1 \cdot crim + \beta_2 \cdot zn + \beta_3 \cdot indus + \beta_4 \cdot nox + \beta_5 \cdot rm + \beta_6 \cdot age + \beta_7 \cdot dis + \beta_8 \cdot rad + \beta_9 \cdot tax + \beta_{10} \cdot ptratio + \beta_{11} \cdot black + \beta_{12} \cdot lstat + \epsilon\]

Where: - \textbf{medv} is the median value of owner-occupied homes
(response variable) - \textbf{crim} through \textbf{lstat} are the
predictor variables - \(\beta_0\) is the intercept - \(\beta_1\) through
\(\beta_{12}\) are the regression coefficients - \(\epsilon\) is the
error term

Note: The variable \textbf{chas} (Charles River dummy variable) is
excluded from this model but could be included in alternative
specifications.

\begin{Shaded}
\begin{Highlighting}[]
\CommentTok{\# Fit the multiple regression model}
\NormalTok{model }\OtherTok{\textless{}{-}} \FunctionTok{lm}\NormalTok{(medv }\SpecialCharTok{\textasciitilde{}}\NormalTok{ crim }\SpecialCharTok{+}\NormalTok{ zn }\SpecialCharTok{+}\NormalTok{ indus }\SpecialCharTok{+}\NormalTok{ nox }\SpecialCharTok{+}\NormalTok{ rm }\SpecialCharTok{+}\NormalTok{ age }\SpecialCharTok{+}\NormalTok{ dis }\SpecialCharTok{+}\NormalTok{ rad }\SpecialCharTok{+}\NormalTok{ tax }\SpecialCharTok{+}\NormalTok{ ptratio }\SpecialCharTok{+}\NormalTok{ black }\SpecialCharTok{+}\NormalTok{ lstat, }
            \AttributeTok{data =}\NormalTok{ boston)}

\CommentTok{\# Display model summary}
\FunctionTok{cat}\NormalTok{(}\StringTok{"Model Summary:}\SpecialCharTok{\textbackslash{}n}\StringTok{"}\NormalTok{)}
\end{Highlighting}
\end{Shaded}

\begin{verbatim}
## Model Summary:
\end{verbatim}

\begin{Shaded}
\begin{Highlighting}[]
\FunctionTok{summary}\NormalTok{(model)}
\end{Highlighting}
\end{Shaded}

\begin{verbatim}
## 
## Call:
## lm(formula = medv ~ crim + zn + indus + nox + rm + age + dis + 
##     rad + tax + ptratio + black + lstat, data = boston)
## 
## Residuals:
##      Min       1Q   Median       3Q      Max 
## -13.3968  -2.8103  -0.6455   1.9141  26.3755 
## 
## Coefficients:
##               Estimate Std. Error t value Pr(>|t|)    
## (Intercept)  36.891960   5.146516   7.168 2.79e-12 ***
## crim         -0.113139   0.033113  -3.417 0.000686 ***
## zn            0.047052   0.013847   3.398 0.000734 ***
## indus         0.040311   0.061707   0.653 0.513889    
## nox         -17.366999   3.851224  -4.509 8.13e-06 ***
## rm            3.850492   0.421402   9.137  < 2e-16 ***
## age           0.002784   0.013309   0.209 0.834407    
## dis          -1.485374   0.201187  -7.383 6.64e-13 ***
## rad           0.328311   0.066542   4.934 1.10e-06 ***
## tax          -0.013756   0.003766  -3.653 0.000287 ***
## ptratio      -0.990958   0.131399  -7.542 2.25e-13 ***
## black         0.009741   0.002706   3.600 0.000351 ***
## lstat        -0.534158   0.051072 -10.459  < 2e-16 ***
## ---
## Signif. codes:  0 '***' 0.001 '**' 0.01 '*' 0.05 '.' 0.1 ' ' 1
## 
## Residual standard error: 4.787 on 493 degrees of freedom
## Multiple R-squared:  0.7355, Adjusted R-squared:  0.7291 
## F-statistic: 114.3 on 12 and 493 DF,  p-value: < 2.2e-16
\end{verbatim}

\section{Results}\label{results}

\begin{Shaded}
\begin{Highlighting}[]
\CommentTok{\# Extract and format coefficients}
\NormalTok{coef\_table }\OtherTok{\textless{}{-}} \FunctionTok{summary}\NormalTok{(model)}\SpecialCharTok{$}\NormalTok{coefficients}
\NormalTok{coef\_df }\OtherTok{\textless{}{-}} \FunctionTok{data.frame}\NormalTok{(}
  \AttributeTok{Variable =} \FunctionTok{rownames}\NormalTok{(coef\_table),}
  \AttributeTok{Estimate =} \FunctionTok{round}\NormalTok{(coef\_table[, }\StringTok{"Estimate"}\NormalTok{], }\DecValTok{4}\NormalTok{),}
  \AttributeTok{Std\_Error =} \FunctionTok{round}\NormalTok{(coef\_table[, }\StringTok{"Std. Error"}\NormalTok{], }\DecValTok{4}\NormalTok{),}
  \AttributeTok{t\_value =} \FunctionTok{round}\NormalTok{(coef\_table[, }\StringTok{"t value"}\NormalTok{], }\DecValTok{3}\NormalTok{),}
  \AttributeTok{p\_value =} \FunctionTok{format.pval}\NormalTok{(coef\_table[, }\StringTok{"Pr(\textgreater{}|t|)"}\NormalTok{], }\AttributeTok{digits =} \DecValTok{3}\NormalTok{)}
\NormalTok{)}

\CommentTok{\# Create bold vector for significant p{-}values}
\NormalTok{p\_numeric }\OtherTok{\textless{}{-}}\NormalTok{ coef\_table[, }\StringTok{"Pr(\textgreater{}|t|)"}\NormalTok{]}
\NormalTok{bold\_cols }\OtherTok{\textless{}{-}}\NormalTok{ p\_numeric }\SpecialCharTok{\textless{}} \FloatTok{0.05}

\FunctionTok{kable}\NormalTok{(coef\_df, }\AttributeTok{caption =} \StringTok{"Regression Coefficients and Significance Tests"}\NormalTok{, }
      \AttributeTok{col.names =} \FunctionTok{c}\NormalTok{(}\StringTok{"Variable"}\NormalTok{, }\StringTok{"Estimate"}\NormalTok{, }\StringTok{"Std. Error"}\NormalTok{, }\StringTok{"t{-}value"}\NormalTok{, }\StringTok{"p{-}value"}\NormalTok{)) }\SpecialCharTok{\%\textgreater{}\%}
  \FunctionTok{kable\_styling}\NormalTok{(}\AttributeTok{latex\_options =} \FunctionTok{c}\NormalTok{(}\StringTok{"striped"}\NormalTok{, }\StringTok{"hold\_position"}\NormalTok{), }\AttributeTok{font\_size =} \DecValTok{9}\NormalTok{) }\SpecialCharTok{\%\textgreater{}\%}
  \FunctionTok{column\_spec}\NormalTok{(}\DecValTok{5}\NormalTok{, }\AttributeTok{bold =}\NormalTok{ bold\_cols)}
\end{Highlighting}
\end{Shaded}

\begingroup\fontsize{9}{11}\selectfont

\begin{longtable}[t]{llrr>{}rl}
\caption{\label{tab:model-coefficients}Regression Coefficients and Significance Tests}\\
\toprule
 & Variable & Estimate & Std. Error & t-value & p-value\\
\midrule
\cellcolor{gray!10}{(Intercept)} & \cellcolor{gray!10}{(Intercept)} & \cellcolor{gray!10}{36.8920} & \cellcolor{gray!10}{5.1465} & \textbf{\cellcolor{gray!10}{7.168}} & \cellcolor{gray!10}{2.79e-12}\\
crim & crim & -0.1131 & 0.0331 & \textbf{-3.417} & 0.000686\\
\cellcolor{gray!10}{zn} & \cellcolor{gray!10}{zn} & \cellcolor{gray!10}{0.0471} & \cellcolor{gray!10}{0.0138} & \textbf{\cellcolor{gray!10}{3.398}} & \cellcolor{gray!10}{0.000734}\\
indus & indus & 0.0403 & 0.0617 & 0.653 & 0.513889\\
\cellcolor{gray!10}{nox} & \cellcolor{gray!10}{nox} & \cellcolor{gray!10}{-17.3670} & \cellcolor{gray!10}{3.8512} & \textbf{\cellcolor{gray!10}{-4.509}} & \cellcolor{gray!10}{8.13e-06}\\
\addlinespace
rm & rm & 3.8505 & 0.4214 & \textbf{9.137} & < 2e-16\\
\cellcolor{gray!10}{age} & \cellcolor{gray!10}{age} & \cellcolor{gray!10}{0.0028} & \cellcolor{gray!10}{0.0133} & \cellcolor{gray!10}{0.209} & \cellcolor{gray!10}{0.834407}\\
dis & dis & -1.4854 & 0.2012 & \textbf{-7.383} & 6.64e-13\\
\cellcolor{gray!10}{rad} & \cellcolor{gray!10}{rad} & \cellcolor{gray!10}{0.3283} & \cellcolor{gray!10}{0.0665} & \textbf{\cellcolor{gray!10}{4.934}} & \cellcolor{gray!10}{1.10e-06}\\
tax & tax & -0.0138 & 0.0038 & \textbf{-3.653} & 0.000287\\
\addlinespace
\cellcolor{gray!10}{ptratio} & \cellcolor{gray!10}{ptratio} & \cellcolor{gray!10}{-0.9910} & \cellcolor{gray!10}{0.1314} & \textbf{\cellcolor{gray!10}{-7.542}} & \cellcolor{gray!10}{2.25e-13}\\
black & black & 0.0097 & 0.0027 & \textbf{3.600} & 0.000351\\
\cellcolor{gray!10}{lstat} & \cellcolor{gray!10}{lstat} & \cellcolor{gray!10}{-0.5342} & \cellcolor{gray!10}{0.0511} & \textbf{\cellcolor{gray!10}{-10.459}} & \cellcolor{gray!10}{< 2e-16}\\
\bottomrule
\end{longtable}
\endgroup{}

\begin{Shaded}
\begin{Highlighting}[]
\CommentTok{\# Extract model fit statistics}
\NormalTok{r\_squared }\OtherTok{\textless{}{-}} \FunctionTok{summary}\NormalTok{(model)}\SpecialCharTok{$}\NormalTok{r.squared}
\NormalTok{adj\_r\_squared }\OtherTok{\textless{}{-}} \FunctionTok{summary}\NormalTok{(model)}\SpecialCharTok{$}\NormalTok{adj.r.squared}
\NormalTok{f\_statistic }\OtherTok{\textless{}{-}} \FunctionTok{summary}\NormalTok{(model)}\SpecialCharTok{$}\NormalTok{fstatistic[}\DecValTok{1}\NormalTok{]}
\NormalTok{f\_pvalue }\OtherTok{\textless{}{-}} \FunctionTok{pf}\NormalTok{(}\FunctionTok{summary}\NormalTok{(model)}\SpecialCharTok{$}\NormalTok{fstatistic[}\DecValTok{1}\NormalTok{], }
               \FunctionTok{summary}\NormalTok{(model)}\SpecialCharTok{$}\NormalTok{fstatistic[}\DecValTok{2}\NormalTok{], }
               \FunctionTok{summary}\NormalTok{(model)}\SpecialCharTok{$}\NormalTok{fstatistic[}\DecValTok{3}\NormalTok{], }
               \AttributeTok{lower.tail =} \ConstantTok{FALSE}\NormalTok{)}

\FunctionTok{cat}\NormalTok{(}\StringTok{"}\SpecialCharTok{\textbackslash{}n}\StringTok{Model Fit Statistics:}\SpecialCharTok{\textbackslash{}n}\StringTok{"}\NormalTok{)}
\end{Highlighting}
\end{Shaded}

\begin{verbatim}
## 
## Model Fit Statistics:
\end{verbatim}

\begin{Shaded}
\begin{Highlighting}[]
\FunctionTok{cat}\NormalTok{(}\StringTok{"R{-}squared:"}\NormalTok{, }\FunctionTok{round}\NormalTok{(r\_squared, }\DecValTok{4}\NormalTok{), }\StringTok{"}\SpecialCharTok{\textbackslash{}n}\StringTok{"}\NormalTok{)}
\end{Highlighting}
\end{Shaded}

\begin{verbatim}
## R-squared: 0.7355
\end{verbatim}

\begin{Shaded}
\begin{Highlighting}[]
\FunctionTok{cat}\NormalTok{(}\StringTok{"Adjusted R{-}squared:"}\NormalTok{, }\FunctionTok{round}\NormalTok{(adj\_r\_squared, }\DecValTok{4}\NormalTok{), }\StringTok{"}\SpecialCharTok{\textbackslash{}n}\StringTok{"}\NormalTok{)}
\end{Highlighting}
\end{Shaded}

\begin{verbatim}
## Adjusted R-squared: 0.7291
\end{verbatim}

\begin{Shaded}
\begin{Highlighting}[]
\FunctionTok{cat}\NormalTok{(}\StringTok{"F{-}statistic:"}\NormalTok{, }\FunctionTok{round}\NormalTok{(f\_statistic, }\DecValTok{2}\NormalTok{), }\StringTok{"}\SpecialCharTok{\textbackslash{}n}\StringTok{"}\NormalTok{)}
\end{Highlighting}
\end{Shaded}

\begin{verbatim}
## F-statistic: 114.25
\end{verbatim}

\begin{Shaded}
\begin{Highlighting}[]
\FunctionTok{cat}\NormalTok{(}\StringTok{"F{-}statistic p{-}value:"}\NormalTok{, }\FunctionTok{format.pval}\NormalTok{(f\_pvalue, }\AttributeTok{digits =} \DecValTok{3}\NormalTok{), }\StringTok{"}\SpecialCharTok{\textbackslash{}n}\StringTok{"}\NormalTok{)}
\end{Highlighting}
\end{Shaded}

\begin{verbatim}
## F-statistic p-value: <2e-16
\end{verbatim}

\begin{Shaded}
\begin{Highlighting}[]
\FunctionTok{cat}\NormalTok{(}\StringTok{"Residual standard error:"}\NormalTok{, }\FunctionTok{round}\NormalTok{(}\FunctionTok{summary}\NormalTok{(model)}\SpecialCharTok{$}\NormalTok{sigma, }\DecValTok{2}\NormalTok{), }\StringTok{"on"}\NormalTok{, }\FunctionTok{summary}\NormalTok{(model)}\SpecialCharTok{$}\NormalTok{df[}\DecValTok{2}\NormalTok{], }\StringTok{"degrees of freedom}\SpecialCharTok{\textbackslash{}n}\StringTok{"}\NormalTok{)}
\end{Highlighting}
\end{Shaded}

\begin{verbatim}
## Residual standard error: 4.79 on 493 degrees of freedom
\end{verbatim}

\subsection{Interpretation}\label{interpretation}

The model explains approximately 73.6\% of the variance in median home
values (R² = 0.7355). The adjusted R² of 0.7291 accounts for the number
of predictors in the model.

The F-statistic (114.25) with a p-value of \textless2e-16 indicates that
the model is statistically significant overall, meaning that at least
one of the predictor variables has a significant relationship with
median home values.

\section{Model Diagnostics}\label{model-diagnostics}

\subsection{Residual Analysis}\label{residual-analysis}

\begin{Shaded}
\begin{Highlighting}[]
\CommentTok{\# Create fitted vs actual plot}
\NormalTok{fitted\_values }\OtherTok{\textless{}{-}} \FunctionTok{fitted}\NormalTok{(model)}
\NormalTok{actual\_values }\OtherTok{\textless{}{-}}\NormalTok{ boston}\SpecialCharTok{$}\NormalTok{medv}

\NormalTok{p1 }\OtherTok{\textless{}{-}} \FunctionTok{ggplot}\NormalTok{(}\FunctionTok{data.frame}\NormalTok{(}\AttributeTok{Fitted =}\NormalTok{ fitted\_values, }\AttributeTok{Actual =}\NormalTok{ actual\_values), }
             \FunctionTok{aes}\NormalTok{(}\AttributeTok{x =}\NormalTok{ Fitted, }\AttributeTok{y =}\NormalTok{ Actual)) }\SpecialCharTok{+}
  \FunctionTok{geom\_point}\NormalTok{(}\AttributeTok{alpha =} \FloatTok{0.6}\NormalTok{, }\AttributeTok{color =} \StringTok{"steelblue"}\NormalTok{) }\SpecialCharTok{+}
  \FunctionTok{geom\_abline}\NormalTok{(}\AttributeTok{intercept =} \DecValTok{0}\NormalTok{, }\AttributeTok{slope =} \DecValTok{1}\NormalTok{, }\AttributeTok{color =} \StringTok{"red"}\NormalTok{, }\AttributeTok{linetype =} \StringTok{"dashed"}\NormalTok{, }\AttributeTok{linewidth =} \DecValTok{1}\NormalTok{) }\SpecialCharTok{+}
  \FunctionTok{labs}\NormalTok{(}\AttributeTok{title =} \StringTok{"Fitted vs Actual Values"}\NormalTok{,}
       \AttributeTok{x =} \StringTok{"Fitted Values (Predicted medv)"}\NormalTok{,}
       \AttributeTok{y =} \StringTok{"Actual Values (medv)"}\NormalTok{) }\SpecialCharTok{+}
  \FunctionTok{theme\_minimal}\NormalTok{() }\SpecialCharTok{+}
  \FunctionTok{theme}\NormalTok{(}\AttributeTok{plot.title =} \FunctionTok{element\_text}\NormalTok{(}\AttributeTok{hjust =} \FloatTok{0.5}\NormalTok{, }\AttributeTok{face =} \StringTok{"bold"}\NormalTok{))}

\FunctionTok{print}\NormalTok{(p1)}
\end{Highlighting}
\end{Shaded}

\begin{center}\includegraphics{regression_analysis_files/figure-latex/fitted-actual-1} \end{center}

\begin{Shaded}
\begin{Highlighting}[]
\CommentTok{\# Residuals vs fitted values}
\NormalTok{residuals }\OtherTok{\textless{}{-}} \FunctionTok{resid}\NormalTok{(model)}

\NormalTok{p2 }\OtherTok{\textless{}{-}} \FunctionTok{ggplot}\NormalTok{(}\FunctionTok{data.frame}\NormalTok{(}\AttributeTok{Fitted =}\NormalTok{ fitted\_values, }\AttributeTok{Residuals =}\NormalTok{ residuals), }
             \FunctionTok{aes}\NormalTok{(}\AttributeTok{x =}\NormalTok{ Fitted, }\AttributeTok{y =}\NormalTok{ Residuals)) }\SpecialCharTok{+}
  \FunctionTok{geom\_point}\NormalTok{(}\AttributeTok{alpha =} \FloatTok{0.6}\NormalTok{, }\AttributeTok{color =} \StringTok{"steelblue"}\NormalTok{) }\SpecialCharTok{+}
  \FunctionTok{geom\_hline}\NormalTok{(}\AttributeTok{yintercept =} \DecValTok{0}\NormalTok{, }\AttributeTok{color =} \StringTok{"red"}\NormalTok{, }\AttributeTok{linetype =} \StringTok{"dashed"}\NormalTok{, }\AttributeTok{linewidth =} \DecValTok{1}\NormalTok{) }\SpecialCharTok{+}
  \FunctionTok{geom\_smooth}\NormalTok{(}\AttributeTok{method =} \StringTok{"loess"}\NormalTok{, }\AttributeTok{se =} \ConstantTok{TRUE}\NormalTok{, }\AttributeTok{color =} \StringTok{"darkgreen"}\NormalTok{, }\AttributeTok{linewidth =} \FloatTok{0.8}\NormalTok{) }\SpecialCharTok{+}
  \FunctionTok{labs}\NormalTok{(}\AttributeTok{title =} \StringTok{"Residuals vs Fitted Values"}\NormalTok{,}
       \AttributeTok{x =} \StringTok{"Fitted Values"}\NormalTok{,}
       \AttributeTok{y =} \StringTok{"Residuals"}\NormalTok{) }\SpecialCharTok{+}
  \FunctionTok{theme\_minimal}\NormalTok{() }\SpecialCharTok{+}
  \FunctionTok{theme}\NormalTok{(}\AttributeTok{plot.title =} \FunctionTok{element\_text}\NormalTok{(}\AttributeTok{hjust =} \FloatTok{0.5}\NormalTok{, }\AttributeTok{face =} \StringTok{"bold"}\NormalTok{))}

\FunctionTok{print}\NormalTok{(p2)}
\end{Highlighting}
\end{Shaded}

\begin{center}\includegraphics{regression_analysis_files/figure-latex/residual-plot-1} \end{center}

\begin{Shaded}
\begin{Highlighting}[]
\CommentTok{\# Q{-}Q plot for normality of residuals}
\NormalTok{p3 }\OtherTok{\textless{}{-}} \FunctionTok{ggplot}\NormalTok{(}\FunctionTok{data.frame}\NormalTok{(}\AttributeTok{Residuals =}\NormalTok{ residuals), }\FunctionTok{aes}\NormalTok{(}\AttributeTok{sample =}\NormalTok{ Residuals)) }\SpecialCharTok{+}
  \FunctionTok{stat\_qq}\NormalTok{(}\AttributeTok{alpha =} \FloatTok{0.6}\NormalTok{, }\AttributeTok{color =} \StringTok{"steelblue"}\NormalTok{) }\SpecialCharTok{+}
  \FunctionTok{stat\_qq\_line}\NormalTok{(}\AttributeTok{color =} \StringTok{"red"}\NormalTok{, }\AttributeTok{linetype =} \StringTok{"dashed"}\NormalTok{, }\AttributeTok{linewidth =} \DecValTok{1}\NormalTok{) }\SpecialCharTok{+}
  \FunctionTok{labs}\NormalTok{(}\AttributeTok{title =} \StringTok{"Q{-}Q Plot of Residuals"}\NormalTok{,}
       \AttributeTok{x =} \StringTok{"Theoretical Quantiles"}\NormalTok{,}
       \AttributeTok{y =} \StringTok{"Sample Quantiles"}\NormalTok{) }\SpecialCharTok{+}
  \FunctionTok{theme\_minimal}\NormalTok{() }\SpecialCharTok{+}
  \FunctionTok{theme}\NormalTok{(}\AttributeTok{plot.title =} \FunctionTok{element\_text}\NormalTok{(}\AttributeTok{hjust =} \FloatTok{0.5}\NormalTok{, }\AttributeTok{face =} \StringTok{"bold"}\NormalTok{))}

\FunctionTok{print}\NormalTok{(p3)}
\end{Highlighting}
\end{Shaded}

\begin{center}\includegraphics{regression_analysis_files/figure-latex/qq-plot-1} \end{center}

\begin{Shaded}
\begin{Highlighting}[]
\CommentTok{\# Residuals vs selected predictors}
\NormalTok{p4 }\OtherTok{\textless{}{-}} \FunctionTok{ggplot}\NormalTok{(}\FunctionTok{data.frame}\NormalTok{(}\AttributeTok{rm =}\NormalTok{ boston}\SpecialCharTok{$}\NormalTok{rm, }\AttributeTok{Residuals =}\NormalTok{ residuals), }
             \FunctionTok{aes}\NormalTok{(}\AttributeTok{x =}\NormalTok{ rm, }\AttributeTok{y =}\NormalTok{ Residuals)) }\SpecialCharTok{+}
  \FunctionTok{geom\_point}\NormalTok{(}\AttributeTok{alpha =} \FloatTok{0.6}\NormalTok{, }\AttributeTok{color =} \StringTok{"steelblue"}\NormalTok{) }\SpecialCharTok{+}
  \FunctionTok{geom\_hline}\NormalTok{(}\AttributeTok{yintercept =} \DecValTok{0}\NormalTok{, }\AttributeTok{color =} \StringTok{"red"}\NormalTok{, }\AttributeTok{linetype =} \StringTok{"dashed"}\NormalTok{, }\AttributeTok{linewidth =} \DecValTok{1}\NormalTok{) }\SpecialCharTok{+}
  \FunctionTok{geom\_smooth}\NormalTok{(}\AttributeTok{method =} \StringTok{"loess"}\NormalTok{, }\AttributeTok{se =} \ConstantTok{TRUE}\NormalTok{, }\AttributeTok{color =} \StringTok{"darkgreen"}\NormalTok{, }\AttributeTok{linewidth =} \FloatTok{0.8}\NormalTok{) }\SpecialCharTok{+}
  \FunctionTok{labs}\NormalTok{(}\AttributeTok{title =} \StringTok{"Residuals vs Number of Rooms (rm)"}\NormalTok{,}
       \AttributeTok{x =} \StringTok{"Average Number of Rooms"}\NormalTok{,}
       \AttributeTok{y =} \StringTok{"Residuals"}\NormalTok{) }\SpecialCharTok{+}
  \FunctionTok{theme\_minimal}\NormalTok{() }\SpecialCharTok{+}
  \FunctionTok{theme}\NormalTok{(}\AttributeTok{plot.title =} \FunctionTok{element\_text}\NormalTok{(}\AttributeTok{hjust =} \FloatTok{0.5}\NormalTok{, }\AttributeTok{face =} \StringTok{"bold"}\NormalTok{))}

\FunctionTok{print}\NormalTok{(p4)}
\end{Highlighting}
\end{Shaded}

\begin{center}\includegraphics{regression_analysis_files/figure-latex/residuals-vs-predictors-1} \end{center}

\begin{Shaded}
\begin{Highlighting}[]
\NormalTok{p5 }\OtherTok{\textless{}{-}} \FunctionTok{ggplot}\NormalTok{(}\FunctionTok{data.frame}\NormalTok{(}\AttributeTok{lstat =}\NormalTok{ boston}\SpecialCharTok{$}\NormalTok{lstat, }\AttributeTok{Residuals =}\NormalTok{ residuals), }
             \FunctionTok{aes}\NormalTok{(}\AttributeTok{x =}\NormalTok{ lstat, }\AttributeTok{y =}\NormalTok{ Residuals)) }\SpecialCharTok{+}
  \FunctionTok{geom\_point}\NormalTok{(}\AttributeTok{alpha =} \FloatTok{0.6}\NormalTok{, }\AttributeTok{color =} \StringTok{"steelblue"}\NormalTok{) }\SpecialCharTok{+}
  \FunctionTok{geom\_hline}\NormalTok{(}\AttributeTok{yintercept =} \DecValTok{0}\NormalTok{, }\AttributeTok{color =} \StringTok{"red"}\NormalTok{, }\AttributeTok{linetype =} \StringTok{"dashed"}\NormalTok{, }\AttributeTok{linewidth =} \DecValTok{1}\NormalTok{) }\SpecialCharTok{+}
  \FunctionTok{geom\_smooth}\NormalTok{(}\AttributeTok{method =} \StringTok{"loess"}\NormalTok{, }\AttributeTok{se =} \ConstantTok{TRUE}\NormalTok{, }\AttributeTok{color =} \StringTok{"darkgreen"}\NormalTok{, }\AttributeTok{linewidth =} \FloatTok{0.8}\NormalTok{) }\SpecialCharTok{+}
  \FunctionTok{labs}\NormalTok{(}\AttributeTok{title =} \StringTok{"Residuals vs Lower Status Population \% (lstat)"}\NormalTok{,}
       \AttributeTok{x =} \StringTok{"Lower Status Population (\%)"}\NormalTok{,}
       \AttributeTok{y =} \StringTok{"Residuals"}\NormalTok{) }\SpecialCharTok{+}
  \FunctionTok{theme\_minimal}\NormalTok{() }\SpecialCharTok{+}
  \FunctionTok{theme}\NormalTok{(}\AttributeTok{plot.title =} \FunctionTok{element\_text}\NormalTok{(}\AttributeTok{hjust =} \FloatTok{0.5}\NormalTok{, }\AttributeTok{face =} \StringTok{"bold"}\NormalTok{))}

\FunctionTok{print}\NormalTok{(p5)}
\end{Highlighting}
\end{Shaded}

\begin{center}\includegraphics{regression_analysis_files/figure-latex/residuals-vs-predictors-2} \end{center}

\subsection{Diagnostic Interpretation}\label{diagnostic-interpretation}

The diagnostic plots help assess the regression assumptions:

\begin{enumerate}
\def\labelenumi{\arabic{enumi}.}
\item
  \textbf{Fitted vs Actual Plot}: Shows how well the model predicts the
  actual values. Points should cluster around the diagonal line (red
  dashed).
\item
  \textbf{Residual Plot}: Tests for homoscedasticity (constant
  variance). The residuals should be randomly scattered around zero with
  no clear patterns. The green smooth line helps identify trends.
\item
  \textbf{Q-Q Plot}: Tests for normality of residuals. Points should
  follow the red diagonal line if residuals are normally distributed.
\item
  \textbf{Residuals vs Predictors}: Helps identify if there are patterns
  in residuals related to specific predictors, which might indicate
  non-linear relationships or missing terms.
\end{enumerate}

\section{Visualizations}\label{visualizations}

\subsection{Scatter Plots: Predictors vs
Response}\label{scatter-plots-predictors-vs-response}

\begin{Shaded}
\begin{Highlighting}[]
\CommentTok{\# Scatter plots of key predictors vs medv}
\NormalTok{p6 }\OtherTok{\textless{}{-}} \FunctionTok{ggplot}\NormalTok{(boston, }\FunctionTok{aes}\NormalTok{(}\AttributeTok{x =}\NormalTok{ rm, }\AttributeTok{y =}\NormalTok{ medv)) }\SpecialCharTok{+}
  \FunctionTok{geom\_point}\NormalTok{(}\AttributeTok{alpha =} \FloatTok{0.6}\NormalTok{, }\AttributeTok{color =} \StringTok{"steelblue"}\NormalTok{) }\SpecialCharTok{+}
  \FunctionTok{geom\_smooth}\NormalTok{(}\AttributeTok{method =} \StringTok{"lm"}\NormalTok{, }\AttributeTok{se =} \ConstantTok{TRUE}\NormalTok{, }\AttributeTok{color =} \StringTok{"red"}\NormalTok{, }\AttributeTok{linewidth =} \DecValTok{1}\NormalTok{) }\SpecialCharTok{+}
  \FunctionTok{labs}\NormalTok{(}\AttributeTok{title =} \StringTok{"Average Number of Rooms vs Median Home Value"}\NormalTok{,}
       \AttributeTok{x =} \StringTok{"Average Number of Rooms"}\NormalTok{,}
       \AttributeTok{y =} \StringTok{"Median Home Value ($1000s)"}\NormalTok{) }\SpecialCharTok{+}
  \FunctionTok{theme\_minimal}\NormalTok{() }\SpecialCharTok{+}
  \FunctionTok{theme}\NormalTok{(}\AttributeTok{plot.title =} \FunctionTok{element\_text}\NormalTok{(}\AttributeTok{hjust =} \FloatTok{0.5}\NormalTok{, }\AttributeTok{face =} \StringTok{"bold"}\NormalTok{))}

\FunctionTok{print}\NormalTok{(p6)}
\end{Highlighting}
\end{Shaded}

\begin{center}\includegraphics{regression_analysis_files/figure-latex/scatter-plots-1} \end{center}

\begin{Shaded}
\begin{Highlighting}[]
\NormalTok{p7 }\OtherTok{\textless{}{-}} \FunctionTok{ggplot}\NormalTok{(boston, }\FunctionTok{aes}\NormalTok{(}\AttributeTok{x =}\NormalTok{ lstat, }\AttributeTok{y =}\NormalTok{ medv)) }\SpecialCharTok{+}
  \FunctionTok{geom\_point}\NormalTok{(}\AttributeTok{alpha =} \FloatTok{0.6}\NormalTok{, }\AttributeTok{color =} \StringTok{"steelblue"}\NormalTok{) }\SpecialCharTok{+}
  \FunctionTok{geom\_smooth}\NormalTok{(}\AttributeTok{method =} \StringTok{"lm"}\NormalTok{, }\AttributeTok{se =} \ConstantTok{TRUE}\NormalTok{, }\AttributeTok{color =} \StringTok{"red"}\NormalTok{, }\AttributeTok{linewidth =} \DecValTok{1}\NormalTok{) }\SpecialCharTok{+}
  \FunctionTok{labs}\NormalTok{(}\AttributeTok{title =} \StringTok{"Lower Status Population \% vs Median Home Value"}\NormalTok{,}
       \AttributeTok{x =} \StringTok{"Lower Status Population (\%)"}\NormalTok{,}
       \AttributeTok{y =} \StringTok{"Median Home Value ($1000s)"}\NormalTok{) }\SpecialCharTok{+}
  \FunctionTok{theme\_minimal}\NormalTok{() }\SpecialCharTok{+}
  \FunctionTok{theme}\NormalTok{(}\AttributeTok{plot.title =} \FunctionTok{element\_text}\NormalTok{(}\AttributeTok{hjust =} \FloatTok{0.5}\NormalTok{, }\AttributeTok{face =} \StringTok{"bold"}\NormalTok{))}

\FunctionTok{print}\NormalTok{(p7)}
\end{Highlighting}
\end{Shaded}

\begin{center}\includegraphics{regression_analysis_files/figure-latex/scatter-plots-2} \end{center}

\begin{Shaded}
\begin{Highlighting}[]
\NormalTok{p8 }\OtherTok{\textless{}{-}} \FunctionTok{ggplot}\NormalTok{(boston, }\FunctionTok{aes}\NormalTok{(}\AttributeTok{x =}\NormalTok{ nox, }\AttributeTok{y =}\NormalTok{ medv)) }\SpecialCharTok{+}
  \FunctionTok{geom\_point}\NormalTok{(}\AttributeTok{alpha =} \FloatTok{0.6}\NormalTok{, }\AttributeTok{color =} \StringTok{"steelblue"}\NormalTok{) }\SpecialCharTok{+}
  \FunctionTok{geom\_smooth}\NormalTok{(}\AttributeTok{method =} \StringTok{"lm"}\NormalTok{, }\AttributeTok{se =} \ConstantTok{TRUE}\NormalTok{, }\AttributeTok{color =} \StringTok{"red"}\NormalTok{, }\AttributeTok{linewidth =} \DecValTok{1}\NormalTok{) }\SpecialCharTok{+}
  \FunctionTok{labs}\NormalTok{(}\AttributeTok{title =} \StringTok{"Nitric Oxide Concentration vs Median Home Value"}\NormalTok{,}
       \AttributeTok{x =} \StringTok{"Nitric Oxide Concentration (parts per 10 million)"}\NormalTok{,}
       \AttributeTok{y =} \StringTok{"Median Home Value ($1000s)"}\NormalTok{) }\SpecialCharTok{+}
  \FunctionTok{theme\_minimal}\NormalTok{() }\SpecialCharTok{+}
  \FunctionTok{theme}\NormalTok{(}\AttributeTok{plot.title =} \FunctionTok{element\_text}\NormalTok{(}\AttributeTok{hjust =} \FloatTok{0.5}\NormalTok{, }\AttributeTok{face =} \StringTok{"bold"}\NormalTok{))}

\FunctionTok{print}\NormalTok{(p8)}
\end{Highlighting}
\end{Shaded}

\begin{center}\includegraphics{regression_analysis_files/figure-latex/scatter-plots-3} \end{center}

\begin{Shaded}
\begin{Highlighting}[]
\NormalTok{p9 }\OtherTok{\textless{}{-}} \FunctionTok{ggplot}\NormalTok{(boston, }\FunctionTok{aes}\NormalTok{(}\AttributeTok{x =}\NormalTok{ crim, }\AttributeTok{y =}\NormalTok{ medv)) }\SpecialCharTok{+}
  \FunctionTok{geom\_point}\NormalTok{(}\AttributeTok{alpha =} \FloatTok{0.6}\NormalTok{, }\AttributeTok{color =} \StringTok{"steelblue"}\NormalTok{) }\SpecialCharTok{+}
  \FunctionTok{geom\_smooth}\NormalTok{(}\AttributeTok{method =} \StringTok{"lm"}\NormalTok{, }\AttributeTok{se =} \ConstantTok{TRUE}\NormalTok{, }\AttributeTok{color =} \StringTok{"red"}\NormalTok{, }\AttributeTok{linewidth =} \DecValTok{1}\NormalTok{) }\SpecialCharTok{+}
  \FunctionTok{labs}\NormalTok{(}\AttributeTok{title =} \StringTok{"Crime Rate vs Median Home Value"}\NormalTok{,}
       \AttributeTok{x =} \StringTok{"Per Capita Crime Rate"}\NormalTok{,}
       \AttributeTok{y =} \StringTok{"Median Home Value ($1000s)"}\NormalTok{) }\SpecialCharTok{+}
  \FunctionTok{theme\_minimal}\NormalTok{() }\SpecialCharTok{+}
  \FunctionTok{theme}\NormalTok{(}\AttributeTok{plot.title =} \FunctionTok{element\_text}\NormalTok{(}\AttributeTok{hjust =} \FloatTok{0.5}\NormalTok{, }\AttributeTok{face =} \StringTok{"bold"}\NormalTok{))}

\FunctionTok{print}\NormalTok{(p9)}
\end{Highlighting}
\end{Shaded}

\begin{center}\includegraphics{regression_analysis_files/figure-latex/scatter-plots-4} \end{center}

\section{Appendix: Variable
Definitions}\label{appendix-variable-definitions}

The following table provides definitions for all variables in the Boston
Housing Dataset:

\begin{Shaded}
\begin{Highlighting}[]
\CommentTok{\# Create variable definitions table}
\NormalTok{variable\_definitions }\OtherTok{\textless{}{-}} \FunctionTok{data.frame}\NormalTok{(}
  \AttributeTok{Variable =} \FunctionTok{c}\NormalTok{(}\StringTok{"crim"}\NormalTok{, }\StringTok{"zn"}\NormalTok{, }\StringTok{"indus"}\NormalTok{, }\StringTok{"chas"}\NormalTok{, }\StringTok{"nox"}\NormalTok{, }\StringTok{"rm"}\NormalTok{, }\StringTok{"age"}\NormalTok{, }\StringTok{"dis"}\NormalTok{, }
               \StringTok{"rad"}\NormalTok{, }\StringTok{"tax"}\NormalTok{, }\StringTok{"ptratio"}\NormalTok{, }\StringTok{"black"}\NormalTok{, }\StringTok{"lstat"}\NormalTok{, }\StringTok{"medv"}\NormalTok{),}
  \AttributeTok{Description =} \FunctionTok{c}\NormalTok{(}
    \StringTok{"Per capita crime rate by town"}\NormalTok{,}
    \StringTok{"Proportion of residential land zoned for lots over 25,000 sq.ft."}\NormalTok{,}
    \StringTok{"Proportion of non{-}retail business acres per town"}\NormalTok{,}
    \StringTok{"Charles River dummy variable (1 if tract bounds river; 0 otherwise)"}\NormalTok{,}
    \StringTok{"Nitric oxides concentration (parts per 10 million)"}\NormalTok{,}
    \StringTok{"Average number of rooms per dwelling"}\NormalTok{,}
    \StringTok{"Proportion of owner{-}occupied units built prior to 1940"}\NormalTok{,}
    \StringTok{"Weighted distances to five Boston employment centres"}\NormalTok{,}
    \StringTok{"Index of accessibility to radial highways"}\NormalTok{,}
    \StringTok{"Full{-}value property{-}tax rate per $10,000"}\NormalTok{,}
    \StringTok{"Pupil{-}teacher ratio by town"}\NormalTok{,}
    \StringTok{"1000(Bk {-} 0.63)\^{}2 where Bk is the proportion of blacks by town"}\NormalTok{,}
    \StringTok{"Lower status of the population (percent)"}\NormalTok{,}
    \StringTok{"Median value of owner{-}occupied homes in $1000s"}
\NormalTok{  ),}
  \AttributeTok{Units\_Scale =} \FunctionTok{c}\NormalTok{(}
    \StringTok{"Continuous"}\NormalTok{,}
    \StringTok{"Proportion (0{-}100)"}\NormalTok{,}
    \StringTok{"Proportion (0{-}100)"}\NormalTok{,}
    \StringTok{"Binary (0 or 1)"}\NormalTok{,}
    \StringTok{"Parts per 10 million"}\NormalTok{,}
    \StringTok{"Average number"}\NormalTok{,}
    \StringTok{"Proportion (0{-}100)"}\NormalTok{,}
    \StringTok{"Weighted distance"}\NormalTok{,}
    \StringTok{"Index"}\NormalTok{,}
    \StringTok{"Rate per $10,000"}\NormalTok{,}
    \StringTok{"Ratio"}\NormalTok{,}
    \StringTok{"Transformed proportion"}\NormalTok{,}
    \StringTok{"Percent"}\NormalTok{,}
    \StringTok{"$1000s"}
\NormalTok{  )}
\NormalTok{)}

\FunctionTok{kable}\NormalTok{(variable\_definitions, }\AttributeTok{caption =} \StringTok{"Variable Definitions for Boston Housing Dataset"}\NormalTok{,}
      \AttributeTok{col.names =} \FunctionTok{c}\NormalTok{(}\StringTok{"Variable"}\NormalTok{, }\StringTok{"Description"}\NormalTok{, }\StringTok{"Units/Scale"}\NormalTok{)) }\SpecialCharTok{\%\textgreater{}\%}
  \FunctionTok{kable\_styling}\NormalTok{(}\AttributeTok{latex\_options =} \FunctionTok{c}\NormalTok{(}\StringTok{"striped"}\NormalTok{, }\StringTok{"hold\_position"}\NormalTok{), }\AttributeTok{font\_size =} \DecValTok{9}\NormalTok{) }\SpecialCharTok{\%\textgreater{}\%}
  \FunctionTok{column\_spec}\NormalTok{(}\DecValTok{1}\NormalTok{, }\AttributeTok{bold =} \ConstantTok{TRUE}\NormalTok{) }\SpecialCharTok{\%\textgreater{}\%}
  \FunctionTok{column\_spec}\NormalTok{(}\DecValTok{2}\NormalTok{, }\AttributeTok{width =} \StringTok{"8cm"}\NormalTok{) }\SpecialCharTok{\%\textgreater{}\%}
  \FunctionTok{column\_spec}\NormalTok{(}\DecValTok{3}\NormalTok{, }\AttributeTok{width =} \StringTok{"3cm"}\NormalTok{)}
\end{Highlighting}
\end{Shaded}

\begingroup\fontsize{9}{11}\selectfont

\begin{longtable}[t]{>{}l>{\raggedright\arraybackslash}p{8cm}>{\raggedright\arraybackslash}p{3cm}}
\caption{\label{tab:variable-definitions}Variable Definitions for Boston Housing Dataset}\\
\toprule
Variable & Description & Units/Scale\\
\midrule
\textbf{\cellcolor{gray!10}{crim}} & \cellcolor{gray!10}{Per capita crime rate by town} & \cellcolor{gray!10}{Continuous}\\
\textbf{zn} & Proportion of residential land zoned for lots over 25,000 sq.ft. & Proportion (0-100)\\
\textbf{\cellcolor{gray!10}{indus}} & \cellcolor{gray!10}{Proportion of non-retail business acres per town} & \cellcolor{gray!10}{Proportion (0-100)}\\
\textbf{chas} & Charles River dummy variable (1 if tract bounds river; 0 otherwise) & Binary (0 or 1)\\
\textbf{\cellcolor{gray!10}{nox}} & \cellcolor{gray!10}{Nitric oxides concentration (parts per 10 million)} & \cellcolor{gray!10}{Parts per 10 million}\\
\addlinespace
\textbf{rm} & Average number of rooms per dwelling & Average number\\
\textbf{\cellcolor{gray!10}{age}} & \cellcolor{gray!10}{Proportion of owner-occupied units built prior to 1940} & \cellcolor{gray!10}{Proportion (0-100)}\\
\textbf{dis} & Weighted distances to five Boston employment centres & Weighted distance\\
\textbf{\cellcolor{gray!10}{rad}} & \cellcolor{gray!10}{Index of accessibility to radial highways} & \cellcolor{gray!10}{Index}\\
\textbf{tax} & Full-value property-tax rate per \$10,000 & Rate per \$10,000\\
\addlinespace
\textbf{\cellcolor{gray!10}{ptratio}} & \cellcolor{gray!10}{Pupil-teacher ratio by town} & \cellcolor{gray!10}{Ratio}\\
\textbf{black} & 1000(Bk - 0.63)\textasciicircum{}2 where Bk is the proportion of blacks by town & Transformed proportion\\
\textbf{\cellcolor{gray!10}{lstat}} & \cellcolor{gray!10}{Lower status of the population (percent)} & \cellcolor{gray!10}{Percent}\\
\textbf{medv} & Median value of owner-occupied homes in \$1000s & \$1000s\\
\bottomrule
\end{longtable}
\endgroup{}

\end{document}
